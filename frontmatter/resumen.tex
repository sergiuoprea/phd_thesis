\chapter{Resumen}

La atención a las personas dependientes (por razones de envejecimiento, accidentes, discapacidades o enfermedades) es una de las líneas de investigación prioritarias para los países europeos, tal y como se recoge en los objetivos del marco Horizonte 2020. Con el fin de minimizar el coste y la intrusividad de las terapias para el cuidado y la rehabilitación, se desea que tales cuidados sean administrados en el hogar del paciente. La solución natural para este entorno es una plataforma robótica móvil en interiores.

Esta plataforma robótica para el cuidado en el hogar necesita resolver hasta cierto punto un conjunto de problemas que se encuentran en la intersección de múltiples disciplinas, por ejemplo, la visión por computador, el aprendizaje por computadora y la robótica. En esa encrucijada, uno de los retos más notables (y en el que nos centraremos) es la comprensión de la escena: el robot necesita entender el entorno desestructurado y dinámico en el que navega y los objetos con los que puede interactuar.

Para lograr una comprensión completa de la escena, se deben realizar varias tareas. En esta tesis nos centraremos en tres de ellas: reconocimiento de la clase de objetos o \emph{object recognition}, segmentación semántica o \emph{semantic segmentation} y predicción de la estabilidad del agarra de los objetos (también denominado \emph{grasp stability prediction}). La primera se refiere al proceso de categorizar un objeto de acuerdo a un conjunto de clases (por ejemplo, silla, cama o almohada); la segunda va un nivel más allá de la categorización de objetos y tiene como objetivo proporcionar un etiquetado denso por píxel de cada objeto en una imagen; la tercera consiste en determinar si un objeto que ha sido agarrado por una mano robótica está en una configuración estable o si va a caer o deslizarse.

Esta tesis presenta contribuciones a la resolución de estas tres tareas utilizando el aprendizaje profundo (\emph{deep learning}) como la metodología principal para resolver estos problemas de reconocimiento, segmentación y predicción. Todas estas soluciones comparten una observación central: todas se basan en datos tridimensionales para aprovechar esa dimensión adicional y su disposición espacial. Las cuatro contribuciones principales de esta tesis son: en primer lugar, mostramos un conjunto de arquitecturas y representaciones de datos para la clasificación de objetos 3D utilizando nubes de puntos; en segundo lugar, llevamos a cabo una extensa revisión del estado del arte de los conjuntos de datos y métodos de segmentación semántica; en tercer lugar, introducimos un novedoso conjunto de datos sintéticos y a gran escala para la resolución conjunta de diversos problemas de robótica y visión; por último, proponemos un método y una representación alternativa para tratar con los sensores táctiles y aprender a predecir la estabilidad de agarre.
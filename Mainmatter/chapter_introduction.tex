\chapter{Introduction}
\label{cha:introduction}

\begin{chapterabstract}
This first chapter introduces the main topic of this thesis. The chapter is organized into five different sections: Section \ref{cha:introduction:sec:motivation} establishes the framework for the research activity proposed in this thesis; Section \ref{cha:introduction:sec:approach} introduces the proposal developed during the thesis; Section \ref{cha:introduction:sec:contributions} states the contributions of this work; Section \ref{cha:introduction:sec:papers} provides a list of co-authored papers that were written during this thesis; finally, Section \ref{cha:introduction:sec:structure} details the structure of the document.
\end{chapterabstract}

\minitoc

\clearpage

\section{Introduction and Motivation}
\label{cha:introduction:sec:motivation}

Nowadays, one of the most important challenges of developed countries is..


 The general goal of this thesis is...

\section{Approach}
\label{cha:introduction:sec:approach}

This work focuses on a subset of the problems that we stated in the previous section from a learning-based point of view:

\begin{itemize}
    \item bla
    \item bla
    \item and bla.
\end{itemize}


\section{Contributions}
\label{cha:introduction:sec:contributions}

As we already stated, this work concentrates on pushing forward... 

\begin{itemize}
    \item bla.
    \item bla.
    \item and bla.
\end{itemize}

\section{Co-Authored Papers}
\label{cha:introduction:sec:papers}

This thesis is the result of continuous effort throughout the last years. Such efforts have sometimes crystallized in form of journal publications, conference talks, and poster presentations. A significant part of this thesis consists of extracts from the following co-authored publications.

\subsection{Chapter ...}

\begin{itemize}
  \item Cite paper %\fullcite{...}
\end{itemize}


\subsection{Other}

During the years spent working on the main topics of this thesis, several collaborations and side works were carried out that also were published either as journal papers, conference proceedings, or preprints. Those works, although not strictly related to the content of this thesis, helped in various ways: exchanging ideas that later inspired other concepts, sparking collaborations, and also expanding the knowledge of other interesting areas of research.

\begin{itemize}
  \item Cite collaborations.
\end{itemize}

\section{Thesis Structure}
\label{cha:introduction:sec:structure}

This thesis is structured as follows... 